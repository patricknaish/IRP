
\documentclass[12pt,journal,compsoc]{IEEEtran}


\ifCLASSINFOpdf
  \usepackage[pdftex]{graphicx}
  \DeclareGraphicsExtensions{.pdf,.jpeg,.png}
\else

\fi

\usepackage{array}
\usepackage{hyperref}
\usepackage{url}


% correct bad hyphenation here
%\hyphenation{op-tical net-works semi-conduc-tor}


\begin{document}

\title{Investigating the feasibility and worth of migrating legacy systems}
\author{Patrick~Naish~(\texttt{pn3g10@soton.ac.uk})}


\IEEEcompsoctitleabstractindextext{%
\begin{abstract}
The abstract goes here.
\end{abstract}}

% Note that keywords are not normally used for peerreview papers.
%\begin{IEEEkeywords}
%Computer Society, IEEEtran, journal, \LaTeX, paper, template.
%\end{IEEEkeywords}}

\maketitle

\IEEEdisplaynotcompsoctitleabstractindextext

\IEEEpeerreviewmaketitle

\section{Introduction}
\label{sec:introduction}

\IEEEPARstart{T}{his} project will be an investigation into the existing and proposed tools and methodologies for migrating mainframe (or similar legacy system) applications to run under more modern systems, such as on PCs, distributed systems and/or web services. Assessing the usefulness of such tools and methodologies is important for making decisions on whether to attempt to recoup previously invested time and resources from existing systems, or invest again in a newer, more flexible version of said system. Therefore, the questions this project aims to answer are whether sufficient tools exist to facilitate code migration, whether there are standard\cite{Almonaies2010} (or at least widely accepted) methodologies in place for doing so, and whether such migrations turn out to be genuinely beneficial for the related parties.

\section{Problem background}
\label{sec:background}

\subsection{Differences between mainframe and modern system architectures}
\label{subsec:differences}

\subsection{Factors prompting/necessitating migration}
\label{subsec:factors}

\subsection{Example case studies}
\label{subsec:examples}

\section{Approaches to system migration}
\label{sec:approaches}

\subsection{End-system structures}
\label{subsec:structures}

\subsubsection{Service-Oriented Architectures}
\label{subsubsec:soa}

\subsubsection{Object-Oriented Platforms}
\label{subsubsec:oop}

\subsection{Tools for code analysis}
\label{subsec:toolsanalysis}

\subsubsection{SoftAudit and complexity metrics}
\label{subsubsec:softaudit}

\subsubsection{COBAudit}
\label{subsubsec:cobaudit}

\subsection{Tools for migration}
\label{subsec:toolsmigration}

\subsubsection{COB2WEB}
\label{subsubsec:cob2web}

\subsubsection{COBRedo}
\label{subsubsec:cobredo}

\subsubsection{COBStrip}
\label{subsubsec:cobstrip}

\subsubsection{COBWrap}
\label{subsubsec:cobwrap}

\subsubsection{COBLink}
\label{subsubsec:coblink}

\subsubsection{etc}

\subsection{Techniques}
\label{subsec:techniques}

\subsubsection{Service-Oriented Migration and Reuse Technique (SMART)}
\label{subsubsec:smart}

\subsubsection{Service-Oriented Software Reengineering Methodology (SoSR)}
\label{subsubsec:sosr}

\subsubsection{Ubiquitous Web Applications Design Framework (UWA)}
\label{subsubsec:uwa}

\subsubsection{etc}

\subsection{Comparison of approaches}
\label{subsec:comparison}

\section{Issues which must be addressed}
\label{sec:issues}

\subsection{Cost-effectiveness}
\label{subsec:costeffectiveness}

\subsection{Maintainability}
\label{subsec:maintainability}

\subsection{Risk}
\label{subsec:risk}

\subsection{Automation}
\label{subsec:automation}

\subsection{Testing}
\label{subsec:testing}

\section{Conclusion}
\label{sec:conclusion}

\appendices
%\section{Proof of the First Zonklar Equation}
%Appendix one text goes here.

% you can choose not to have a title for an appendix
% if you want by leaving the argument blank
%\section{}
%Appendix two text goes here.


% use section* for acknowledgement
\ifCLASSOPTIONcompsoc
  % The Computer Society usually uses the plural form
  %\section*{Acknowledgments}
\else
  % regular IEEE prefers the singular form
  \section*{Acknowledgment}
\fi

%The authors would like to thank...


\ifCLASSOPTIONcaptionsoff
  \newpage
\fi

\bibliographystyle{IEEEtran}
\bibliography{IEEEabrv,references}

% biography section
%
% If you have an EPS/PDF photo (graphicx package needed) extra braces are
% needed around the contents of the optional argument to biography to prevent
% the LaTeX parser from getting confused when it sees the complicated
% \includegraphics command within an optional argument. (You could create
% your own custom macro containing the \includegraphics command to make things
% simpler here.)
%\begin{biography}[{\includegraphics[width=1in,height=1.25in,clip,keepaspectratio]{mshell}}]{Michael Shell}
% or if you just want to reserve a space for a photo:

%\begin{IEEEbiography}{Michael Shell}
%Biography text here.
%\end{IEEEbiography}

% if you will not have a photo at all:
%\begin{IEEEbiographynophoto}{John Doe}
%Biography text here.
%\end{IEEEbiographynophoto}

% insert where needed to balance the two columns on the last page with
% biographies
%\newpage

%\begin{IEEEbiographynophoto}{Jane Doe}
%Biography text here.
%\end{IEEEbiographynophoto}

% You can push biographies down or up by placing
% a \vfill before or after them. The appropriate
% use of \vfill depends on what kind of text is
% on the last page and whether or not the columns
% are being equalized.

%\vfill

% Can be used to pull up biographies so that the bottom of the last one
% is flush with the other column.
%\enlargethispage{-5in}



% that's all folks
\end{document}


